\documentclass{article}
\usepackage{amsmath, amsfonts, amssymb}
\usepackage{xcolor}
\usepackage{xepersian}
\settextfont{BNazanin}
\title{سوالات دست گرمی}
\author{امین احدی}
\begin{document}
	\maketitle
	 \begin{enumerate}
	 	\item[\lr{Q1} ] برنامه‌ای بنویسید که در آن سه عدد صحیح را از کاربر بگیرد و پس از دریافت اعداد آنها را به ترتیب صعودی در صفحه نمایش چاپ کند.
	 \\	برنامه‌ی خود را در فایل \lr{Q1.py} ذخیره کنید.
	 	\item[\lr{Q2}] برنامه‌ای بنویسید که سه عدد صحیح را از کاربر دریافت کند. پس از دریافت اعداد بررسی کند که آیا  اعداد وارد شده تشکیل مثلث می‌دهند. 
	 \\-	چنانچه اعداد وارد شده قابلیت ساخت مثلث را ندارند در خروجی چاپ کنید 
	 	\lr{Not Triangle }
	 \\-	 چنانچه با اعداد وارد شده می‌توان مثلثی ساخت در خروجی چاپ کنید 
	 	\lr{Yeeeees!}
	 \\	برنامه‌ی خود را با نام \lr{Q2.py} نمایید.
	 	
		\item[\lr{Q3}] برنامه‌ای بنویسید که مراحل سوال دو را تکرار کند، اما این بار بررسی کنید که آیا مثلث ساخته شده با اعداد ورودی  تشکیل یک مثلث قایم الزاویه می‌دهند یا خیر 
		\\-  چنانچه اعداد ورودی تشکیل مثلث نمی‌دهند در صفحه نمایش چاپ کنید 
		\lr{Not Triangle}
	\\-	چنانچه اعداد وارد شده تشکیل مثلث می‌دهند، اما این مثلث یک مثلث قایم‌الزاویه نیست، در صفحه نمایش چاپ کنید
		\lr{You have a triangle, but not right triangle}
	\\-	و در پایان چنانچه اعداد وارد شده تشکیل یک مثلث قایم‌الزاویه می‌دهند در صفحه نمایش چاپ کنید
		 \lr{Hooooraaaa :)}
	\\	فایل خود را با نام \lr{Q3.py} ذخیره نمایید.
		
		
		
	
		\textcolor{red}{نکته! }\\
		برنامه‌های خود را در یک پوشه (دایرکتوری) با نام خانوادگی خود در صفحه اصلی لپ تاپ \lr{Desktop} به صورت یک فایل زیپ شده ذخیره نمایید.
		
	 	
	 
	 \end{enumerate}
\end{document}